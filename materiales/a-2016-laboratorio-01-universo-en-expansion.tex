\documentclass[a4paper,12pt]{article}
\usepackage[spanish]{babel}
\hyphenation{co-rres-pon-dien-te}
%\usepackage[latin1]{inputenc}
\usepackage[utf8]{inputenc}
\usepackage[T1]{fontenc}
\usepackage{graphicx}
\usepackage[pdftex,colorlinks=true, pdfstartview=FitH, linkcolor=blue,
citecolor=blue, urlcolor=blue, pdfpagemode=UseOutlines, pdfauthor={H. Asorey},
pdftitle={Física 1A - Guía 02}, pdfkeywords={formé, unidades}]{hyperref}
\usepackage[adobe-utopia]{mathdesign}

\hoffset -1.23cm
\textwidth 16.5cm
\voffset -2.0cm
\textheight 26.0cm

%----------------------------------------------------------------
\begin{document}
\title{
{\normalsize{Universidad Nacional de Río Negro - Profesorados de Física y Química}}\\
Física I A \\ Guía de Laboratorio - Universo en Expansión}
\author{Asorey - Cutsaimanis}
\date{2016}
\maketitle

\section*{Materiales}

\begin{itemize}
	\item Globo
	\item Marcador
	\item Hilo
	\item Regla
\end{itemize}

\section*{Procedimiento}

\subsection*{Preparación}
\begin{enumerate}
	\item Inflar el globo hasta que quede más o menos como una esfera pequeña
	\item Dibujar 6 puntos distribuidos al azar sobre toda la superficie del
		globo y numerarlos del 1 al 6.
	\item Elegir un punto cualquiera al azar. Este punto será nuestro origen de
		coordenadas.
\end{enumerate}

\subsection*{Parte 1: La expansión}
\begin{enumerate}
	\item Usando la regla, medir el tamaño de un punto (usar siempre el mismo)
	\item Usando el hilo, y considerando la menor distancia posible sobre la
		superficie curva del globo, medir las distancias entre el punto elegido
		y cada uno de los otros cinco puntos. 
	\item Inflar un poco el globo, y volver al punto 4). Iterar este
		procedimiento 5 veces, armando una tabla con las mediciones obtenidas.
	\item Elegir otro punto como origen de coordenadas y repetir las mediciones
		de los puntos 5) y 6).
\end{enumerate}

\subsection*{Parte 2: El corrimiento al rojo}
\begin{enumerate}
	\item Desinflar el globo hasta que quede del tamaño original. Dibujar en
		algún lado una onda tipo seno.
	\item Medir la longitud de onda y anotarla en una tabla.
	\item Inflar un poco el globo, y volver al punto 9). Iterar este
		procedimiento 5 veces.
\end{enumerate}

\subsection*{Parte 3: Evolución de la densidad de materia}
\begin{enumerate}
	\item Desinflar el globo hasta que quede del tamaño original. Ahora dibuje
		24 puntos adicionales de manera que la cantidad total sobre la
		superficie sea de 30 puntos. 
	\item Diseñar un método para medir la densidad superficial de puntos, es
		decir, una magnitud escalar que precise la cantidad promedio de puntos
		(30) por unidad de superficie. Una forma sería, por ejemplo, determinar
		la superficie del globo midiendo el perímetro, para luego calcular el
		radio $r$. Luego, con el radio y suponiendo que se puede aproximar al
		globo como una esfera, la superficie del globo será $S=4 \pi r^2$. Con
		este valor, estimar la ``densidad superficial de puntos'' (puntos
		cm$^{-2}$) y anotarla en una tabla.
	\item Inflar un poco el globo, y volver al punto anterior. Iterar este
		procedimiento 5 veces.
\end{enumerate}

\section*{Preguntas para pensar (lista no excluyente)}
\begin{enumerate}
	\item ¿Hay alguna diferencia en la forma en que se separan los puntos
		lejanos y los puntos cercanos sobre la superficie del globo?  
	\item ¿Podemos pensar que existe un punto ``central'' sobre la superficie
		del globo?
	\item ¿Qué sucedió cuando usamos otro punto de referencia para el origen?
	\item ¿Cómo ``evoluciona'' la densidad superficial con el radio del globo?
	\item ¿Cómo ``evoluciona'' la longitud de la onda con la expansión del
		globo?
\end{enumerate}
\end{document}
