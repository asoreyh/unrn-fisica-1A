\documentclass[a4paper,12pt]{article}
\usepackage[spanish]{babel}
\usepackage[utf8]{inputenc}
\usepackage[T1]{fontenc}
\usepackage{graphicx}
\usepackage{mathabx}
\usepackage[pdftex,colorlinks=true, pdfstartview=FitH,
linkcolor=blue,citecolor=blue, urlcolor=blue, pdfpagemode=UseOutlines,
pdfauthor={H. Asorey},pdftitle={Física 1A - 2012 - Recuperatorio}]{hyperref}
\usepackage[adobe-utopia]{mathdesign}

\hoffset -1.23cm
\textwidth 16.5cm
\voffset -2.0cm
\textheight 25.0cm

\begin{document}
\pagestyle{empty}
Nombre y Apellido: \ldots\ldots\ldots\ldots\ldots\ldots\ldots\ldots\ldots\ldots
Página 1 de \ldots\ldots\ldots \begin{center}
{\Large {\bf{Física I A}}} \\
\bigskip
{\large {\bf{Recuperatorio}}} \\ {\bf{23 Junio 2012}} \\
\end{center}

\section*{Consideraciones generales}

\begin{itemize}
\item Dispone de {\bf{tres horas}} para completar el examen.
\item Comience cada problema en una hoja separada. Coloque su nombre y
numere todas las hojas.
\item {\bf{El parcial debe ser escrito en lapicera. Si se equivoca, TACHE. No
use corrector ni goma.}}
\item {\bf{Para aprobar este recuperatorio es necesario aprobar tanto las
preguntas de concepto como los problemas de desarrollo.}}
\end{itemize}

\section*{Datos útiles (Utilice sólo los que considere necesarios)}
 
\indent $G = 6.67 \times 10^{-11}$\,m$^3$\,kg$^{-1}$\,s$^{-2}$;\\
\indent $H_0 = 71 $\,km\,s$^{-1}$\,Mpc$^{-1}$;\\
\indent $c = 3 \times 10^{8}$\,m\,s$^{-1}$;\\
\indent $1\mathrm{\,UA} = 1.5 \times 10^{11}$\,m;\\
\indent $1$\,año$=3.16 \times 10^{7}$\,s;\\
\indent $\sigma = 5.67 \times 10^{-8}$\,W\,m$^{-2}$\,K$^{-4}$;\\
\indent $M_\odot = 1.988\times10^{30}$\,kg;\\
\indent $L_\odot = 3.84\times10^{26}$\,W;\\
\indent $R_\odot=695000$\,km;\\
\indent $M_{\leftmoon} = 7.35\times10^{22}$\,kg;\\
\indent $R_{\leftmoon} = 1.74$\,km.

\section*{Preguntas conceptuales}

\begin{enumerate}

\item Describa las características que debe poseer alguna forma de energía para
ser considerada como energía potencia. Justifique su respuesta y de un ejemplo
cualitativo. 

\item En el sistema planetario planteado por Ptolomeo analice la validez de las
leyes de Kepler.

\item Comente la validez de la siguiente escena de vista en la película ``Sr. y
Sra. Smith''. En la escena, Angelina Jollie toma una itaca (Arma de gran
tamaño) con una sola mano y dispara. El arma se dispara y ella queda de pie y
se arregla el cabello.

\item Si un astrónomo de una galaxia muy lejana observa a nuestra galaxia. ¿La
verá alejarse o acercarse? ¿Las líneas espectrales se correrían al rojo o al
azul? Justifique.


\end{enumerate}

\section*{Problemas}

\subsection*{PRIMER PARCIAL}

\begin{enumerate}

\item Suponga que se encuentra en la Luna y necesita calcular la variación de
energía potencial de un cuerpo al ser desplazado desde la superficie lunar
hasta una altura $h$. Diga en qué casos usaría una expresión de la forma $mgh$,
y en cuáles usaría la expresión general. Luego de un ejemplo cuantitativo de
cada caso y realice todos los cálculos y utilice todos los argumentos que
considere necesarios para justificar claramente su respuesta.

\end{enumerate}

\subsection*{SEGUNDO PARCIAL}

\begin{enumerate}

\item Calcule el radio que debería tener una lamparita de $100$\,W si se comportara como un cuerpo negro. Luego calcule el flujo de energía sobre una pared que se encuentra a $2$\,m de distancia de la lámpara.

\item ¿Cuánta masa debe convertirse en energía para mantener funcionando una
heladera que consume $200$\,W durante 100 años?

\item Calcule la mínima distancia al Sol para un cometa que se mueve siguiendo
una órbita con elipticidad $\epsilon=0.1$ si el período orbital es de 100 años.

\item ¿A qué distancia se encuentra una galaxia si se observa que la línea azul
del hidrógeno ($\lambda_e=434$\,nm) tiene una longitud de onda
$\lambda_o=477.4$\,nm.

\end{enumerate}
\end{document}

%%%%
