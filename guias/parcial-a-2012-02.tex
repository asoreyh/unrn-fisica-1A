\documentclass[a4paper,12pt]{article}
\usepackage[spanish]{babel}
\usepackage[utf8]{inputenc}
\usepackage[T1]{fontenc}
\usepackage{graphicx}
\usepackage[pdftex,colorlinks=true, pdfstartview=FitH,
linkcolor=blue,citecolor=blue, urlcolor=blue, pdfpagemode=UseOutlines,
pdfauthor={H. Asorey},pdftitle={Física 1A - 2012 - Segundo Parcial}]{hyperref}
\usepackage[adobe-utopia]{mathdesign}

\hoffset -1.23cm
\textwidth 16.5cm
\voffset -2.0cm
\textheight 25.0cm

\begin{document}
\pagestyle{empty}
Nombre y Apellido: \ldots\ldots\ldots\ldots\ldots\ldots\ldots\ldots\ldots\ldots
Página 1 de \ldots\ldots\ldots \begin{center}
{\Large {\bf{Física I A}}} \\
\bigskip
{\large {\bf{Segundo Parcial}}} \\ {\bf{23 Junio 2012}} \\
\end{center}

\section*{Consideraciones generales}

\begin{itemize}
\item Dispone de {\bf{tres horas}} para completar el examen.
\item Comience cada problema en una hoja separada. Coloque su nombre y
numere todas las hojas.
\item {\bf{El parcial debe ser escrito en lapicera. Si se equivoca, TACHE. No use corrector ni goma.}}
\item {\bf{Para aprobar este parcial es necesario aprobar tanto las
preguntas de concepto como los problemas de desarrollo.}}
\end{itemize}

\section*{Datos útiles}
 
\indent $G = 6.67 \times 10^{-11}$\,m$^3$\,kg$^{-1}$\,s$^{-2}$;\\
\indent $H_0 = 71 $\,km\,s$^{-1}$\,Mpc$^{-1}$;\\
\indent $b = 2.9 $\,mm\,K;\\
\indent $c = 3 \times 10^{8}$\,m\,s$^{-1}$;\\
\indent $1\mathrm{\,UA} = 1.5 \times 10^{11}$\,m;\\
\indent $\sigma = 5.67 \times 10^{-8}$\,W\,m$^{-2}$\,K$^{-4}$;\\
\indent $M_\odot = 1.988\times10^{30}$\,kg;\\
\indent $L_\odot = 3.84\times10^{26}$\,W;\\
\indent $R_\odot=695000$\,km


\section*{Preguntas de concepto}

\begin{enumerate}
\item Compare los sistemas planetarios propuesto por Ptolomeo, Copérnico y 
Tycho Brahe.
\item Analice si el fragmento siguiente es físicamente válido en su totalidad,
e identifique aquello que no lo sea. Justifique.

En el film {\emph{Vengador del futuro}}, se muestra que el hombre conquistó el
planeta Marte y hay una ciudad construida allí. Para llegar a esta ciudad, con
naves similares a las que llegó el hombre a la Luna, tardan una semana y lo
hacen en cualquier momento del año. Durante toda la película se observa un
cielo rojo y dos lunas, el sol se muestra de un tamaño considerable, mayor al
que observamos desde la Tierra. Como la temperatura del planeta es muy elevada,
y la atmósfera no está adaptada para el hombre, usan unos trajes pintorescos
para salir al exterior. En el final de la película se pone en funcionamiento
una gran máquina que hace que los seres humanos puedan salir a la superficie
sin traje.  En la última escena se observa un cielo celeste y el sol brilla con
mucha intensidad.

\item Analice con un ejemplo {\bf{cuantitativo}} de la vida cotidiana el 
principio de conservación del impulso.
\end{enumerate}

\section*{Problemas}

\begin{enumerate}

\item{\bf{Allá lejos, 3 puntos}}

Una galaxia muy lejana se aleja de nosotros debido a la expansión del Universo
con velocidad $v=10000$\,km\,s$^{-1}$. 
\begin{enumerate}
\item Explique con sus palabras el mecanismo de expansión. ¿Con que velocidad
un astrónomo de esa galaxia vería alejarse a nuestra galaxia Vía Láctea? 
\item ¿A qué distancia se encuentra esa galaxia de nosotros? Exprese el
resultado en Mpc.
\item Calcule la longitud de onda con la que se observará en la Tierra la
emisión del hidrógeno de $656.3$\,nm.
\end{enumerate}


\item{\bf{Eris, 3 puntos}}

Eris es el planeta (menor) más lejano al Sol, y su órbita es elíptica. En el
punto más cercano al Sol, la distancia entre el planeta y el Sol es de
$33$\,UA, mientras que en el punto más lejano es $99$\,UA. Responda: 
\begin{enumerate}
\item ¿Cuál es el radio mayor $a$ y el radio menor $b$ de la órbita? 
\item ¿Cuánto tiempo tarda Eris en completar una órbita? Exprese el resultado
en años.
\item ¿Cuál es la relación entre la velocidad orbital medida en el perihelio
respecto a la medida en el afelio?
\end{enumerate} 

\item{\bf{Estrella, 3 puntos}}

Se determinó que la línea de máxima emisión de una estrella es
$\lambda_{\max}=600$\,nm, y su masa es $3.73 M_\odot$. Calcule el radio de la
estrella. Discuta los posibles finales para esta estrella en particular.

\item {\bf{Nave interplanetaria, 1 punto}}

Calcule la cantidad de masa que debería convertir en energía para lograr que una
nave de masa $m=1000$\,kg situada a una distancia $r_1=10^{11}$\,m del Sol llegue 
con velocidad límite igual a 0 a una distancia $r_2=10^{12}$\,m del Sol. Suponga 
que los únicos cuerpos son la nave y el Sol.

\end{enumerate}
\end{document}

%%%%
