\documentclass[a4paper,12pt]{article}
\usepackage[spanish]{babel}
\hyphenation{co-rres-pon-dien-te}
%\usepackage[latin1]{inputenc}
\usepackage[utf8]{inputenc}
\usepackage[T1]{fontenc}
\usepackage{graphicx}
\usepackage[pdftex,colorlinks=true, pdfstartview=FitH, linkcolor=blue,
citecolor=blue, urlcolor=blue, pdfpagemode=UseOutlines, pdfauthor={H. Asorey},
pdftitle={Física 1A - Guía 06}]{hyperref}
\usepackage[adobe-utopia]{mathdesign}

\hoffset -1.23cm
\textwidth 16.5cm
\voffset -2.0cm
\textheight 26.0cm

%----------------------------------------------------------------
\begin{document}
\title{
{\normalsize{Universidad Nacional de Río Negro - Profesorados de Física y
Química}}\\ Física I A \\ Guía 07 - Sistema Solar}
\author{Asorey - Cutsaimanis}
\date{2016}
\maketitle

\begin{enumerate}
\setcounter{enumi}{43}      %% Offset en numero de problema

	\item {\bf{Elipses}}
		\begin{enumerate}
			\item Verifique que el círculo, cuya expresión es $x^2 + y^2 = r^2$,
				pertenece a la familia de las elipses.
			\item La excentricidad $\epsilon$ es una características de las cónicas
				y está dada por \[ \epsilon = \sqrt{1 - \frac{b^2}{a^2}} =
				\frac{\sqrt{a^2-b^2}}{a}\] Utilizando las propiedades de la elipse,
				demuestre que la distancia $f$ entre el centro de la elipse y uno
				de los focos está dada por: \[f=a \, \epsilon\]
			\item Utilizando el {\emph{método del jardinero}}, construya una elipse
				que tenga $a=10$\,cm, y $b=5$\,cm.
		\end{enumerate}
	\item{\bf{Primera ley de Kepler}}.
		Recordemos la primera ley de Kepler: ``{\emph{Todos los planetas se
		desplazan alrededor del Sol describiendo órbitas elípticas, estando el
		Sol situado en uno de sus focos}}'' Utilizando los valores del afelio,
		perihelio y excentricidad de Mercurio, Venus, la Tierra, Urano y
		Plutón, calcule para cada uno de ellos lo siguiente:
		\begin{enumerate}
			\item Los valores de $a$ y $b$ para cada una de las órbitas;
			\item La distancia desde el ``{\emph{otro foco}}'' al Sol.
		\end{enumerate}
	\item{\bf{Satélite geoestacionario}}
		\begin{enumerate}
			\item A partir de la expresión para la velocidad orbital de una
				órbita circular, \[v_O = \frac{2 \pi r}{t},\] usando la tercera
				ley de Kepler demuestre que para el caso circular la velocidad
				orbital vale: \[v_O = \sqrt{\frac{GM}{r}}. \] dónde $G$ es la
				constante de gravitación universal, $M$ es la masa del cuerpo
				central, $r$ es el radio de la órbita y $T$ es el periodo
				orbital.  
			\item Calcule el radio $r$ para la órbita de un satélite
				geo-estacionario ($T=24$\,horas).
		\end{enumerate}
	
	\item{\bf{Los planetas y la tercera ley}}
		Utilizando los períodos orbitales y las distancias al Sol para Venus,
		La Tierra, Marte, Júpiter, Saturno, Urano y Neptuno, verifique la
		tercera ley de Kepler. Para ello, grafique en excel el cuadrado del
		período orbital, medido en días, como función del cubo de la distancia
		al Sol, medida en millones de kilómetros. La pendiente de la recta
		obtenida, $\Delta y / \Delta x$, no es otra que la constante $K$, sólo
		que expresada en otro sistema de unidades.  Para terminar, convierta
		las unidades al sistema métrico internacional y compárelo con el valor
		obtenido en el punto anterior. Justifique las diferencias encontradas.
	
	\item{\bf{Cometa no es barrilete}}
		Un nuevo cometa de masa $m=10^{12}$\,kg fue descubierto en el sistema
		solar. Luego de algunas mediciones, se supo que su órbita es elíptica
		y el perihelio está situado a sólo $10^6$ km del Sol.
		\begin{enumerate}
			\item Calcule la distancia al Sol del afelio sabiendo que el
				período es de 10 años.
			\item ¿Cuáles es el valor de la energía potencial en el perihelio y
				en el afelio?
			\item Usando la segunda ley de Kepler, calcule la relación entre
				las energías cinéticas en el afelio y en el perihelio (ayuda:
				suponga que las áreas barridas son triangulares, $A=\frac{1}{2}
				b \times h$).
		\end{enumerate}
	\item{\bf{Velocidad orbital}} Imagine que un planeta de masa $m=M_\oplus$
		orbita en torno a una estrella de masa $M=6.5 \times 10^{30}$\,kg. a
		una distancia media $r=4.5\times10^8$\,km. Usando las leyes de Kepler,
		calcule el tiempo que requiere el planeta para completar una órbita
		completa y su velocidad orbital media.
	
	\item{\bf{Mercurio, el planeta}}
		La órbita del planeta Mercurio es bastante alongada y posee una de las
		mayores excentricidades del Sistema Solar, siendo sólo superada por
		Plutón. Las distancias al Sol en su perihelio y afelio son:
		Perihelio: $\overline{PE} = 45\,943\,700$\,km; Afelio $\overline{AF} =
		69\,874\,671$\,km, respectivamente.
		\begin{enumerate}
			\item Calcule la excentricidad $\epsilon$ de la órbita y determine
				el valor de los semiejes $a$ y $b$.
			\item Se sabe que el impulso angular de Mercurio en su órbita es $L
				= 8.9585 \times 10^{38}$\,kg\,m$^2$\,s$^{-1}$. Calcule la
				velocidad del planeta en el perihelio y en el afelio.
			\item Usando la Tercera Ley de Kepler, determine el periodo $T$ del
				planeta.
		\end{enumerate}
	\item{\bf{Crónicas mercuriales}}
		Considere un satélite de masa $m_s$ girando en una órbita circular
		alrededor del planeta Mercurio (masa $M_M = 3.3\times 10^{23}$\,kg;
		radio $R_M = 2440$\,km). El satélite se encuentra a una altura tal que
		se verifica:
		\[E_c + E_p = -\left ( \frac{m_s}{2} \right ) \left (6.48 \times
		10^6\right) \mathrm{\ J\ kg}^{-1} \]
		\begin{enumerate}
			\item Calcule la altura del satélite sobre la superficie de
				Mercurio, su velocidad orbital $v_o$, y el tiempo requerido
				para completar una órbita.
			\item Si el satélite es una esfera de radio $r=10$\,m con una
				densidad media $\rho =3.9$\,g\,cm$^{-3}$, calcule su energía
				cinética y potencial.
			\item Calcule la velocidad de escape para un cuerpo que se halle
				sobre la superficie de Mercurio.
		\end{enumerate}
\end{enumerate}
\end{document}
