\documentclass[a4paper,12pt]{article}
\usepackage[spanish]{babel}
%\usepackage[latin1]{inputenc}
\usepackage[utf8]{inputenc}
\usepackage[T1]{fontenc}
\usepackage{graphicx}
\usepackage[pdftex,colorlinks=true, pdfstartview=FitH,
linkcolor=blue,citecolor=blue, urlcolor=blue, pdfpagemode=UseOutlines,
pdfauthor={H. Asorey},pdftitle={Física 1A - 2016 - Segundo Parcial}]{hyperref}
%\usepackage{palatino}
%\usepackage{amssymb}
\usepackage[adobe-utopia]{mathdesign}

\hoffset -2cm
\textwidth 18.7cm
\voffset -2.2cm
\textheight 26.0cm

\begin{document}
\pagestyle{empty}
\begin{center}
{\Large {\bf{Física I A}}} \\
{\large {\bf{Segundo Parcial}}} \\ {\bf{21 Junio 2016}} \\
\end{center}
\vspace{-2em}
\section{Consideraciones generales}

\begin{itemize}
	\item Dispone de {\bf{tres horas}} para completar el examen.
	\item Comience cada problema en una hoja separada. Coloque su nombre y
		numere todas las hojas.  
	\item No utilice lápiz. Todas sus respuestas deben estar escritas en tinta.
		Si se equivoca, no use corrector ni goma, tache con una línea y/o
		aclare, y continúe debajo.
\end{itemize}

\section{Preguntas para pensar}

\begin{enumerate}
	\item Dadas dos estrellas con temperaturas $T_1$ y $T_2 > T_1$ y radios
		$R_1 < R_2$, ¿cuál de las dos será más rojiza?  
	\item Imagine un choque elástico entre dos cuerpos con masas $m_1$ y $m_2$
		con velocidades iniciales $u_1 = - u_2$. Describa el movimiento de los
		cuerpos ($v_1$ y $v_2$) luego de la colisión cuando $m_1 > m_2$ y
		cuando $m_1 \ll m_2$. 
	\item Usando las leyes de Kepler, explique si es posible determinar la
		masa de la Tierra a partir de la distancia a la Luna y del tiempo que
		esta demora en completar una órbita. 
	\item Analice con un ejemplo {\bf{cuantitativo}} de la vida cotidiana el
		principio de conservación de la cantidad de movimiento.
\end{enumerate}

\section*{Problemas}

\begin{enumerate}
	\item{\bf{Eris}} (3 puntos):
		Eris es el planeta (menor) más lejano al Sol, y su órbita es elíptica.
		En el punto más cercano al Sol, la distancia entre el planeta y el Sol
		es de $33$\,UA, mientras que en el punto más lejano es $99$\,UA.
		Responda:
		\begin{enumerate}
			\item ¿Cuál es el radio mayor $a$ y el radio menor $b$ de la
				órbita?
			\item ¿Cuánto tiempo tarda Eris en completar una órbita? Exprese
				el resultado en años.
			\item Calcule la relación entre la velocidad orbital de Eris en el
				perihelio respecto al afelio. 
		\end{enumerate}
	\item{\bf{Estrella}} (3 puntos):
		Se determinó que la línea de máxima emisión de una estrella es
		$\lambda_{\max}=600$\,nm, y su masa es $3.73 M_\odot$. Calcule el
		radio de la estrella. Si se encuentra a 100 años luz de nosotros,
		¿cuál sería el flujo de energía observado en la Tierra?
	\item {\bf{Nave interplanetaria}} (1 puntos):
		Calcule la cantidad de masa que debería convertir en energía para
		lograr que una nave de masa $m=1000$\,kg situada a una distancia
		$r_1=10^{11}$\,m del Sol llegue con velocidad límite igual a 0 a una
		distancia $r_2=10^{12}$\,m del Sol. Suponga que los únicos cuerpos son
		la nave y el Sol.
	\item {\bf{Hot Wheelsi}} (3 puntos):
		Dos autitos de colección {\emph{Hot Wheels}} con masas $m_1=0.1$\,kg y
		$m_2=0.12$\,kg y velocidades $u_1=5$\,m/s y $u_2=-4$\,m/s colisionan
		de frente. Calcule la velocidad final de cada carro $v_1$ y $v_2$ en
		el caso de una colisión elástica, y luego la velocidad final de ambos
		carros unidos $v$ en el caso de una colisión completamente inelástica.
		En este segundo caso, calcule la pérdida de energía cinética.
\end{enumerate}

\noindent {\bf{Datos útiles}}: $G = 6.67 \times 10^{-11}$\,m$^3$\,kg$^{-1}$\,s$^{-2}$;
$b_{\mathrm{Wien}} = 2.9 $\,mm\,K;
$c = 3 \times 10^{8}$\,m\,s$^{-1}$;
$1\mathrm{\,UA} = 1.5 \times 10^{11}$\,m;
$\sigma = 5.67 \times 10^{-8}$\,W\,m$^{-2}$\,K$^{-4}$;
$M_\odot = 1.988\times10^{30}$\,kg;
$1$\,año luz$=9.4161\times 10^{15}$\,m
$L_\odot = 3.84\times10^{26}$\,W;
$R_\odot=695000$\,km
\end{document}

%%%%
