\documentclass[a4paper,12pt]{article}
\usepackage[spanish]{babel}
%\usepackage[latin1]{inputenc}
\usepackage[utf8]{inputenc}
\usepackage[T1]{fontenc}
\usepackage{graphicx}
\usepackage[pdftex,colorlinks=true, pdfstartview=FitH,
linkcolor=blue,citecolor=blue, urlcolor=blue, pdfpagemode=UseOutlines,
pdfauthor={H. Asorey},pdftitle={Física 1A - 2012 - Primer Parcial}]{hyperref}
%\usepackage{palatino}
%\usepackage{amssymb}
\usepackage[adobe-utopia]{mathdesign}

\hoffset -1.23cm
\textwidth 16.5cm
\voffset -2.0cm
\textheight 25.0cm

\begin{document}
\pagestyle{empty}
Nombre y Apellido: \ldots\ldots\ldots\ldots\ldots\ldots\ldots\ldots\ldots\ldots
Página 1 de \ldots\ldots\ldots \begin{center}
{\Large {\bf{Física I A}}} \\
\bigskip
{\large {\bf{Primer Parcial}}} \\ {\bf{04 Mayo 2012}} \\
\end{center}

\section{Consideraciones generales}

\begin{itemize}
\item Dispone de {\bf{tres horas}} para completar el examen.
\item Comience cada problema en una hoja separada. Coloque su nombre y
numere todas las hojas.  
\item {\bf{Para aprobar este parcial es necesario aprobar tanto las
preguntas de concepto como los problemas de desarrollo.}} 
\end{itemize}

\section{Preguntas de concepto}

\begin{enumerate}
\item Indique cuál es la diferencia conceptual entre las expresiones
$E_g=mgh$ y $E_g = -G M m / R$ para la energía potencial gravitatoria.
Indique para que situaciones son válidas cada una de las expresiones. ¿Cuándo usaría una u otra?
Justifique.
\item Utilizando el gráfico de la función de la energía potencial
gravitatoria, describa el proceso del lanzamiento de una sonda espacial
capaz de alejarse desde la superficie de la Tierra hasta el infinito.
(Suponga que no existen otros cuerpos en el Universo más que la Tierra
y la sonda).
\item Describa {\bf{todas}} las transformaciones de energía que están
implicadas en el ascenso del teleférico del Cerro Otto.
\item Anímese y responda: ¿cuál es la energía {\bf{total}} de un cuerpo de masa $m$ que se mueve a velocidad $v$? Justifique.
\end{enumerate}

\section{Problemas}

\begin{enumerate}

\item{\bf{Resortín, 2 puntos}}

Una bola de billar, de radio $r=2$\,cm y densidad
$\rho=2$\,g\,cm$^{-3}$, se encuentra apoyada sobre un resorte de
constante elástica $k=60$\,N\,m$^{-1}$. El resorte y la bola están inicialmente
en equilibrio (la fuerza elástica equilibra a la fuerza de gravedad). Un agente
externo, comprime al resorte una distancia $\Delta x=0.05$\,m, y luego suelta
la bola, la cual es disparada al aire.  Dibuje la situación planteada y luego
calcule:
\begin{enumerate}
\item el peso de la bola de billar;
\item la altura que alcanza la bola de billar;
\item la velocidad inicial de la bola en el momento en la cual sale disparada;
\item la velocidad cuando la bola alcanza una altura de $0.03$\,m.
\end{enumerate}
Datos útiles: $g = 9.8$\,m\,s$^{-2}$; 

\item{\bf{Titán, 4 puntos}}

Titán es el satélite más importante de Saturno, tiene forma esférica con un
radio $R=2500$\,km y una densidad media $\rho=2000$\,kg\,m$^{-3}$.

\begin{enumerate}
\item Determine la masa de Titán;
\item calcule el peso de un astronauta de masa $m=70$\,kg sobre la
superficie de Titán;
\item calcule el radio de la órbita del planeta.
\item dibuje la función de la energía potencial gravitatoria de Titán,
identificando el radio del satélite.
\item calcule la velocidad de escape, y la energía necesaria para
lograr que un cuerpo de masa $m=100$\,kg escape definitivamente de la
atracción gravitatoria de Titán.
\end{enumerate}

Datos útiles: $G = 6.67 \times 10^{-11}$\,N\,m$^2$\,kg$^{-2}$; 

\item {\bf{Sonda interestelar, 4 puntos}}

Se desea construir una sonda, de masa $m=1000$\,kg, que logre escapar
definitivamente de la influencia de la gravedad {\bf{solar}}. La misma
será construida en la Tierra, a una distancia $R=1.5\times10^{11}$\,m
del centro del Sol. 

\begin{enumerate}
\item Haga un dibujo de la situación planteada. 
\item Calcule la velocidad y la energía cinética que la sonda deberá
tener para cumplir con su objetivo.
\item Compare la velocidad calculada en el punto anterior con las
correspondientes velocidades de escape de la Tierra y del Sol.
Justifique claramente su respuesta.
\end{enumerate}
Datos útiles: $G = 6.67 \times 10^{-11}$\,N\,m$^2$\,kg$^{-2}$;
$M_\oplus = 5.97\times10^{24}$\,kg; $R_\oplus=6378$\,km;
$M_\odot = 1.988\times10^{30}$\,kg; $R_\odot=695000$\,km;


\end{enumerate}
\end{document}

%%%%
