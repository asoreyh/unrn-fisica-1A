\documentclass[a4paper,12pt]{article}
\usepackage[spanish]{babel}
\hyphenation{co-rres-pon-dien-te}
%\usepackage[latin1]{inputenc}
\usepackage[utf8]{inputenc}
\usepackage[T1]{fontenc}
\usepackage{graphicx}
\usepackage[pdftex,colorlinks=true, pdfstartview=FitH, linkcolor=blue,
citecolor=blue, urlcolor=blue, pdfpagemode=UseOutlines, pdfauthor={H. Asorey},
pdftitle={Física 1A - Guía 02}, pdfkeywords={formé, unidades}]{hyperref}
\usepackage[adobe-utopia]{mathdesign}

\hoffset -1.23cm
\textwidth 16.5cm
\voffset -2.0cm
\textheight 26.0cm

%----------------------------------------------------------------
\begin{document}
\title{
{\normalsize{Universidad Nacional de Río Negro - Profesorados de Física y Química}}\\
Física I A \\ Guía 03 - Masa, energía y otras yerbas}
\author{Asorey - Cutsaimanis}
\date{2016}
\maketitle

\begin{enumerate}
\setcounter{enumi}{14}      %% Offset en numero de problema

\item {\bf{Masa y Energía}}

A partir de la equivalencia entre masa y energía, 
\begin{equation}
E = m c^2
\end{equation} 
calcule la energía liberada si el contenido de una botella de Coca
Cola\texttrademark de $330$\,mL se convierte en energía. Luego, calcule cuanto
tiempo se mantendría encendida una lampara de $60$\,W utilizando esa energía.
Ayuda: Suponga que la densidad de la Coca Cola es igual a la del agua
($\rho_{\textrm{\tiny{Coca\,Cola}}}={\rho_{\textrm{\tiny{agua}}}}$).

\item {\bf{Luz y energía}}

Calcule la frecuencia, en Hz, y la energía, en eV ($1$\,eV =
$1.602\times10^{-19}$\,J), de un fotón en cada una de las siguientes bandas de
radiación EM:
\begin{enumerate}
\item microondas, $\lambda=10^6$\,nm
\item infrarrojo, $\lambda=10^3$\,nm
\item rojo, $\lambda=680$\,nm
\item amarillo, $\lambda=550$\,nm
\item violeta, $\lambda=400$\,nm
\item ultravioleta, $\lambda=10^2$\,nm
\item X, $\lambda=1$\,nm
\item gamma, $\lambda=10^{-3}$\,nm
\end{enumerate}

\item {\bf{Decaimiento $\mathbf{\beta}$}}

En el decaimiento $\beta$, un neutrón de masa $939.56$\,MeV/$c^2$, decae en un
protón de masa $939.56$\,MeV/$c^2$, un electrón de masa $938.27$\,MeV/$c^2$, y
un neutrino de masa que podemos considerar despreciable. Calcule:
\begin{enumerate}
\item la masa del neutrón, protón y del electrón en kg;
\item la energía máxima disponible en MeV y en J.
\end{enumerate}

$1$\,MeV = $1.602\times10^{-13}$\,J; $m_e=0.511$\,MeV;

\item {\bf{Luminosidad solar}}

Se define a la luminosidad de una estrella como la energía emitida por la
estrella por unidad de tiempo, $L \equiv \Delta E/\Delta t$, y se mide en watts
($1$\,W$=1$\,J\,s$^{-1}$). En particular, la luminosidad solar es $L_\odot =
3.84\times 10^{26}\,W$. Suponiendo que toda la energía se libera en proceso
visto en clase, donde la energía liberada es de $27.2$\,MeV por reacción,
calcule, considerando $\Delta t=1$\,s, 
\begin{enumerate}
\item La cantidad de reacciones que se producen
\item La cantidad de Hidrógeno consumido
\item La cantidad de Helio generado
\item La cantidad de masa que se libera en forma de energía
\end{enumerate}

\item {\bf{Creación de pares}}

Calcule la energía, la frecuencia y la longitud de onda que debe tener un fotón
para ser capaz de crear un par electrón-positrón con $0.200$\,MeV de energía
cinética total.

Datos útiles:\\
$c=3\times10^8$\,m\,s$^{1}$;
$h=6.63\times10^{-34}$\,J\,s$=4.136\times10^{-15}$\,eV\,s

\item {\bf{Sonda espacial}}

NASA está ensayando un nuevo motor experimental para lanzar una sonda al
espacio profundo. El motor funciona gracias a la equivalencia entre masa y
energía y ha probado tener una eficiencia del $75\%$. Si la masa total del
motor más el combustible más la sonda es de $10000$\,kg, calcule la cantidad de
materia que el motor debe convertir en energía para que el cohete pueda escapar
definitivamente de la atracción gravitatoria terrestre.


\end{enumerate}

\end{document}


\item {\bf{Tiempo y distancia de Hubble}}

\begin{enumerate}
\item Determine el tiempo de Hubble, $t_0 = 1 / H_0$, en años ($H_0=71$\,km\,s$^{-1}$\,Mpc$^{-1}$)
\item Determine la distancia de Hubble, $r_0 = c\times t_0 = c / H_0$, en Mpc.
\end{enumerate}

\item {\bf{Densidad crítica}}

Hemos visto que el Universo se encuentra en expansión, y lo hace con una
velocidad que depende de la distancia $h$, relación conocida como la ley de
Hubble:

\[v = H h.\]

De esta manera, si tenemos una esfera de radio $R$ que se expande, la velocidad
a la cual la superficie de la esfera se aleja del centro de la misma es

\[v = H R.\]

Partiendo de este hecho, y recordando la ecuación de la velocidad de escape
obtenida en la guía 02,

\[v_e = \sqrt{\frac{2 G M}{R}} \]

calcularemos la densidad crítica del Universo, es decir, la densidad para la
cual la fuerza de gravedad de la masa contenida sería capaz de detener la
expansión.

Para ello, 

\begin{enumerate}
\item obtenga la expresión para la masa $M$ de una esfera de radio $R$ y volumen $V$ en función de la densidad $\rho$ (ayuda, recuerde $\rho=M/V$);
\item reemplace esta expresión para $M$ en la ecuación para la velocidad de escape;
\item despeje $\rho$;
\item si $\rho$ representa a la densidad crítica, la superficie de la esfera se
aleja del centro con la velocidad de escape. Puesto que a su vez se verifica la
ley de Hubble, tenemos que $v_e = H R$. Reemplace este valor para la velocidad
de escape en la expresión para $\rho$ obtenida en el punto anterior;
\item verifique que el resultado obtenido, 
\[\rho_c = \frac{3 H^2}{8 \pi G}, \]
no depende del radio $R$;
\item finalmente, calcule el valor de $\rho_c$ en kg m$^{-3}$, tomando $H=71$
km s$^{-1}$ Mpc$^{-1}$. ¿A cuántos átomos de Hidrógeno por m$^3$ equivale esa
densidad?  
\end{enumerate}

\item {\bf{Masa y Energía}}

A partir de la equivalencia entre masa y energía, 
\begin{equation}
E = m c^2
\end{equation} 
calcule durante cuanto tiempo se mantendría encendida una lampara de $60$\,W
suponiendo que todo el contenido de una botella de Coca Cola\texttrademark de 330 mL se convierte en energía. Suponga que $\rho_{\textrm{\tiny{Coca\,Cola}}}={\rho_{\textrm{\tiny{agua}}}}$. 

%%%%


