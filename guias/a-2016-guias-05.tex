\documentclass[a4paper,12pt]{article}
\usepackage[spanish]{babel}
\hyphenation{co-rres-pon-dien-te}
%\usepackage[latin1]{inputenc}
\usepackage[utf8]{inputenc}
\usepackage[T1]{fontenc}
\usepackage{graphicx}
\usepackage[pdftex,colorlinks=true, pdfstartview=FitH, linkcolor=blue,
citecolor=blue, urlcolor=blue, pdfpagemode=UseOutlines, pdfauthor={H. Asorey},
pdftitle={Física 1A - Guía 05}]{hyperref}
\usepackage[adobe-utopia]{mathdesign}

\hoffset -1.23cm
\textwidth 16.5cm
\voffset -2.0cm
\textheight 26.0cm

%----------------------------------------------------------------
\begin{document}
\title{
{\normalsize{Universidad Nacional de Río Negro - Profesorados de Física y
Química}}\\ Física I A \\ Guía 05 - Evolución Estelar}
\author{Asorey - Cutsaimanis}
\date{2016}
\maketitle

\begin{enumerate}
\setcounter{enumi}{24}      %% Offset en numero de problema

\item{\bf{Ley de Wien}}

Hallar el valor de $\xi$ utilizando los datos de la siguiente tabla y
compararlo con el valor aceptado, $\xi = 2.897 \times 10^6$\,nm\,K.

\begin{tabular}{ r | r }
$\lambda_{\max}$ [nm] & T [K]\\
\hline
29000 & 100 \\
5750 & 500 \\
1170 & 2500 \\
510 & 5500 \\
280 & 11000
\end{tabular}
(ayuda: realice los cambios de variables apropiados para linealizar la ecuación
y grafique, calculando la pendiente de la recta obtenida).

\item{\bf{Colores}}

Utilice la ley de Wien para determinar $\lambda_{\max}$, la correspondiente
frecuencia $f_{\max}$, y el color de cada una de las siguientes estrellas: 

\begin{tabular}{ l | r }
Nombre & $T$ [K] \\
\hline
Sol & $5777$ \\
Mintaka & $31000$ \\
Betelgeuse & $3400$ \\
Sirio A & $9540$ \\
Rigel & $11000$ \\
$\eta$-Carinae & $39000$
\end{tabular}

\item{\bf{(*) Betelgeuse y Rigel}}

Betelgeuse($\alpha$-Ori) y Rigel ($\beta$-Ori) son las dos estrellas más
brillantes de la constelación de Orión. Sus posiciones se conocen con excelente
precisión, habiéndose medido un paralaje de $5.07\times10^{-3}$\,arcseg para
Betelgeuse y  $4.22\times10^{-3}$\,arcseg para Rigel. Utilizando un bolómetro
en órbita, ha sido posible medir los flujos de energía en la Tierra:
$\mathcal{F}_{\mathrm{Betelgeuse}} = 8.6845$\,W\,m$^{-2}$ y
$\mathcal{F}_{\mathrm{Rigel}} = 3.7819$\,W\,m$^{-2}$.
 
\begin{enumerate}
\item Calcule la distancia de la Tierra a estas dos estrellas, medidas en m,
años-luz y parsecs (recuerde que si el paralaje es $1$\,arcseg la estrella se
encuentra a $1$\,parsec de distancia, y la relación es inversamente
proporcional). 
\item Calcule las luminosidades de Betelgeuse y Rigel. Expresarlas en unidades
de $L_\odot$ y en W.
\item Calcule las masas de Betelgeuse y Rigel ($M_\odot =
1.899\times10^{30}$kg).
\item Utilice las temperaturas de las estrellas dadas en el ejercicio 26 para
estimar los radios de las mismas. 
\item ¿Dentro de que clasificación espectral las colocaría? ¿En que posición
del diagrama H-R las ubicaría? Justifique.
\item Calcule los radios mínimos y máximos de la zona de habitabilidad de cada estrella.
\end{enumerate}

\item{\bf{Observación astronómica}}

Durante el invierno, mirando hacia el Este y a media altura antes de la
medianoche es posible observar la constelación de Scorpio.
La estrella más brillante (Antares) se encuentra a 600 años luz de la Tierra.
Sabiendo que tiene el mismo color que Betelgeuse y que su masa es $M=15.5
M_\odot$, calcule la Luminosidad y el radio de Antares. Luego determine el tamaño mínimo y máximo de la zona de habitabilidad.

\item{\bf{Temperaturas}}

La temperatura superficial del Sol es $T=5777$ K. Entonces:

\begin{enumerate}
\item Utilice la ley de Steffan-Boltzmann para estimar el valor de $L_\odot$ y
compárelo con el valor conocido.
\item En los últimos estadios de su vida, se sabe que el Sol se transformará en
una gigante roja. Suponiendo que la temperatura superficial disminuirá hasta
$T=3200$, calcule cuál deberá ser la luminosidad solar para que la distancia
entre la Tierra y la superficie del Sol hinchado sea mayor de $10^7$ km.
Exprese el resultado en watts y en unidades de la luminosidad solar actual.
\item ¿Cuál será el destino final del Sol? Suponiendo que el objeto resultante
tiene un $90\%$ de la masa solar actual y el radio típico de esos objetos,
calcule la densidad, el valor de $g$ y la velocidad de escape $v_e$ sobre la
superficie del mismo.
\item ¿Cuál sería el nuevo valor de $\lambda_{\max}$ si por alguna razón la
temperatura superficial aumenta hasta $T = 10500$ K? En este caso, y suponiendo
que $L_\odot$ no cambiaría, ¿qué debería pasar con el radio solar?
\end{enumerate}

\item{\bf{(*) Supernova supernueva}}

Cuando una estrella se convierte en supernova, hasta el 1\% de su masa se
libera en forma de energía. De esta energía, el 99\% se libera en forma de
neutrinos y el resto como radiación electromagnética. Imaginemos que Canopus
($\alpha$-Car, F0, $M=8.5\ M_\odot$, $d=310$ años-luz) se convierte en
supernova.
\begin{enumerate}
\item Calcule la cantidad de energía liberada como neutrinos (indetectable).
\item Calcule el flujo de energía electromagnética que se medirá en la Tierra.
Compare este valor con la constante solar,
$\mathcal{F}_\odot=1400$\,W\,m$^{-2}$.
\item El objeto resultante será una estrella de neutrones, con un radio
aproximado de $R=20$\,km. Calcule la densidad, el valor de $g$ y la velocidad
de escape $v_e$ sobre la superficie de la estrella de neutrones.
\item Calcule el radio de Schwarzschild de Canopus. Compárelo con el obtenido
para el Sol.
\end{enumerate}

\item{\bf{Con una estrella en nuestro interior}}

\begin{enumerate}
\item Calcule la Luminosidad $L$ del cuerpo humano ($T_h = 310$\,K) suponiendo que la temperatura del entorno es de $20^\mathrm{o}$C ($T_e=293$\,K). Recuerde que es posible aproximar al cuerpo humano por un cilindro de densidad $d=1010$\,kg m$^{-3}$, con una masa $m=70$\,kg y $h=170$\,cm de altura. Expresar el resultado en W. 
\item ¿De dónde proviene esta energía? ¿Qué importancia tiene este resultado en la vida diaria?
\item Imaginemos ahora que multiplicamos la L obtenida en el punto anterior por un factor $10^{25}$ y situamos a esta nueva estrella a una distancia de $65.2$\,años luz de la Tierra. Calcule:
\begin{enumerate}
\item La Luminosidad en unidades de $L_\odot$.
\item La masa en kg.
\item El radio de la estrella, suponiendo que es esférica y su temperatura efectiva es $T=3400$\,K 
\item ¿En qué zona del diagrama H-R ubicaría usted a esta estrella?
\item ¿Cuál de los tres posibles finales usted espera debe tener
esta estrella? 
\end{enumerate}
\end{enumerate}

\item{\bf{Producción de energía (Optativo)}}

La masa de un núcleo es menor que la suma de las masas de los protones y neutrones que lo componen. Esto se debe a la contribución negativa de la energía de unión, que según la relación $E=mc^2$ corresponde a una masa. Esa diferencia se denomina {\emph{defecto de masa}}:
\[ \Delta m = N m_n + Z m_p - m\]
donde:
\begin{itemize}
\item $m$ es la masa del núcleo
\item $N$ es el número de neutrones (por ende $N=A-Z$, dónde $A$ es el número másico)
\item $Z$ es el número atómico (igual al número de protones)
\item $m_p = 938.3$\,MeV/c$^2$ y $m_n = 939.6$\,MeV/c$^2$ son las masas del protón y del neutrón respectivamente.
\end{itemize}

En este contexto, la energía de ligadura por nucleón queda dada por: 

\[ B = \frac{\Delta m c^2}{A} \]

Calcule el defecto de masa y la energía de ligadura por nucleón de los siguientes átomos:
\begin{enumerate}
\item $^4_2$He (Helio-4, $m=3728.4$\,MeV/c$^2$).
\item $^{56}_{26}$Fe (Hierro-56, $m=52103$\,MeV/c$^2$).
\item $^{208}_{82}$Pb (Plomo-208, $m=193729$\,MeV/c$^2$).
\item $^{40}_{20}$Ca (Calcio-40, $m=37225$\,MeV/c$^2$).
\item $^{41}_{20}$Ca (Calcio-41, $m=38156$\,MeV/c$^2$).
\end{enumerate}

\item {\bf{Atucha (Optativo)}}

La central nuclear Atucha obtiene su energía de la reacción de fisión del isótopo $^{235}_{92}$U (Uranio-235), con una masa de $m=218942$\,MeV/c$^2$.

\begin{enumerate}
\item Determine la cantidad de neutrones del isótopo.
\item Calcule el defecto de masa del Uranio-235 en MeV/c$^2$.
\item Calcule la energía de unión por nucleón.
\end{enumerate}

\end{enumerate}

\end{document}
%%%%
