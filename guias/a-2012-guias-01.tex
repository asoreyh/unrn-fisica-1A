\documentclass[a4paper,12pt]{article}
\usepackage[spanish]{babel}
\hyphenation{co-rres-pon-dien-te}
%\usepackage[latin1]{inputenc}
\usepackage[utf8]{inputenc}
\usepackage[T1]{fontenc}
\usepackage{graphicx}
\usepackage[pdftex,colorlinks=true, pdfstartview=FitH, linkcolor=blue,
citecolor=blue, urlcolor=blue, pdfpagemode=UseOutlines, pdfauthor={H. Asorey},
pdftitle={Física 1A - Guía 01}, pdfkeywords={fermi, unidades}]{hyperref}
\usepackage[adobe-utopia]{mathdesign}

\hoffset -1.23cm
\textwidth 16.5cm
\voffset -2.0cm
\textheight 26.0cm

%----------------------------------------------------------------
\begin{document}
\title{
{\normalsize{Universidad Nacional de Río Negro - Profesorados de Física y Química}}\\
Física I A \\ Guía 01 - Calentando motores\\}
\author{Asorey - Cutsaimanis}
\date{2012}
\maketitle

Los ejercicios marcados con un {\bf{*}} son de entrega obligatoria.

\begin{enumerate}
\setcounter{enumi}{0}      %% Offset en numero de problema

\item Conteste las siguientes preguntas de Fermi:

\begin{enumerate}
\item ¿Cuántas células hay en el cuerpo humano? (Ayuda: Suponga que una célula
es una esfera de radio $r_0=10$\,$\mu$m de diámetro)
\item ¿Cuántos metros de tubos capilares tenemos en el cuerpo? (Ayuda: En el
cuerpo hay 5 litros de sangre y use el valor de $r_0$ del punto anterior).
\item (*) ¿Cuántas pelotas de fútbol son necesarias para cubrir una cancha?
\item (*) ¿Qué potencia eroga un ser humano en reposo?
\item ¿Cuál es la masa del Lago Nahuel Huapi?
\item ¿Cuánta energía hay almacenada en un litro de nafta?
\item ¿Cuántos kilogramos de basura se producen en Bariloche en un año?
\item ¿Cuánta energía consume un televisor en un año?
\item ¿Cuánta gente es posible acomodar, durante una manifestación, en una
plaza llena de gente?

\end{enumerate}

\item (*) La distancia de la Tierra al Sol se denomina {\emph {Unidad Astronómica}} (UA), y su valor es $1$\,UA=$1.5\times10^8$\,km. 

\begin{enumerate}
\item Exprese el valor de $1$\,UA en metros y
milímetros. Escriba cada uno de esos valores en notación decimal, notación
científica, y utilizando los prefijos específicos de los múltiplos del SI que
mejor se adecúen a cada caso (p. ej. $3\times10^{18}$\,m=3\,Em, tres
exámetros).  \item Imagine ahora una esfera de radio $r=1$\,UA. Calcule la
superficie y el volumen de esta esfera para el radio medido en km, m y mm
(trabaje sólo en notación científica).
\item Suponga que llenamos la esfera del punto anterior hasta la mitad con agua
($\rho_{H_2O} = 1.00$\,g\,cm$^{-3}$), y luego la completamos con aceite vegetal
($\rho_{a} = 0.70$\,g\,cm$^{-3}$). Calcule la masa de agua y de aceite
utilizados, expresando el resultado en microgramos.
\item Utilizando el valor de la velocidad de la luz en el vacío $c$ ($c=299 792
458$\,m\,s$^{-1}$), calcule el tiempo requerido por la luz del Sol para
alcanzar la Tierra. Exprese el resultado en minutos. 
\end{enumerate}

\item Repita ahora todos los cálculos del punto anterior pero para una esfera
de radio $r=500$\,$\mu$m. 

\item Trabajemos con la velocidad de la luz. Entonces:

\begin{enumerate}
\item Viajando a la velocidad de la luz, ¿cuánto tiempo se necesita para
recorrer $1$ metro? 
\item El tiempo requerido por la luz para cubrir la distancia Bariloche-Buenos
Aires ($1600$\,km).
\item ¿Cuántos metros recorre la luz en un año? Este valor se conoce como
{\emph{año luz}} y se lo utiliza para expresar {\bf{distancias}} astronómicas.
\item Se entiende al radio de Bohr $a_\infty$ como al radio clásico de un átomo
de Hidrógeno. ¿Cuanto tiempo necesita un fotón para cubrir una distancia igual
a $a_\infty=0.53$\,angstroms? 
\end{enumerate}
%%%%%%%%%%%%%%%%%%%%%%%%%%%%%%%%
\end{enumerate}
\end{document}
%%%%
