\documentclass[a4paper,12pt]{article}
\usepackage[spanish]{babel}
%\usepackage[latin1]{inputenc}
\usepackage[utf8]{inputenc}
\usepackage[T1]{fontenc}
\usepackage{graphicx}
\usepackage[pdftex,colorlinks=true, pdfstartview=FitH,
linkcolor=blue,citecolor=blue, urlcolor=blue, pdfpagemode=UseOutlines,
pdfauthor={H. Asorey},pdftitle={Física 1A - 2012 - Primer Parcial}]{hyperref}
%\usepackage{palatino}
%\usepackage{amssymb}
\usepackage[adobe-utopia]{mathdesign}

\hoffset -1.23cm
\textwidth 16.5cm
\voffset -2.0cm
\textheight 25.0cm

\begin{document}
\pagestyle{empty}
Nombre y Apellido: \ldots\ldots\ldots\ldots\ldots\ldots\ldots\ldots\ldots\ldots
Página 1 de \ldots\ldots\ldots \begin{center}
{\Large {\bf{Física I A}}} \\
\bigskip
{\large {\bf{Recuperatorio Primer Parcial}}} \\ {\bf{23 Junio 2016}} \\
\end{center}

\section{Consideraciones generales}

\begin{itemize}
	\item Dispone de {\bf{tres horas}} para completar el examen.
	\item Comience cada problema en una hoja separada. Coloque su nombre y
		numere todas las hojas.  
	\item No utilice lápiz. Todas sus respuestas deben estar escritas en tinta.
		Si se equivoca, no use corrector ni goma, tache con una línea y/o
		aclare, y continúe debajo.
\end{itemize}

\section{Preguntas para pensar}

\begin{enumerate}
	\item Indique cuál es la diferencia conceptual entre las expresiones
		$$\Delta E_g=m g_\oplus h \qquad \mathrm{y} \qquad \Delta E_g = -G
		M_\oplus m \left (\frac{1}{R_\oplus + h} - \frac1R_\oplus \right )$$
		para la energía potencial gravitatoria.  Indique para que situaciones
		son válidas cada una de las expresiones.  ¿Cuándo usaría una u otra?
		Justifique. 
	\item Si se arrojan dos cuerpos con igual forma pero distintas masas,
		$m_1<m_2$, desde la misma altura, ¿cómo serían los tiempos de llegada
		al piso? Justifique
	\item Describa {\bf{todas}} las transformaciones de energía que están
		implicadas en el despegue de un avión.
\end{enumerate}

\section{Problemas}

\begin{enumerate}
	\item{\bf{Resortín, 3 puntos}}
		Una bola de billar, de radio $r=3$\,cm y densidad
		$\rho=1$\,g\,cm$^{-3}$, se encuentra apoyada sobre un resorte de
		constante elástica $k=50$\,N\,m$^{-1}$. El resorte y la bola están
		inicialmente en equilibrio (la fuerza elástica equilibra a la fuerza de
		gravedad). Un agente externo, comprime al resorte una distancia $\Delta
		x=0.10$\,m, y luego suelta la bola, la cual sale disparada al aire.
		Dibuje la situación planteada y luego calcule:
		\begin{enumerate}
			\item el peso de la bola de billar;
			\item la altura que alcanza la bola de billar;
			\item la velocidad inicial de la bola en el momento en la cual sale
				disparada;
			\item la velocidad cuando la bola alcanza una altura de $0.04$\,m
				respecto a su posición de equilibrio.
		\end{enumerate}
		Datos útiles: $g = 9.8$\,m\,s$^{-2}$;

	\item{\bf{Supertierra, 4 puntos}}
		Acaba de ser descubierto un planeta en los confines del Sistema Solar,
		algo más grande que la Tierra, con masa $M=4 M_\oplus$ y $R=1.2
		R_\oplus$, y se encuentra a una distancia de $r_f=30$\,UA del Sol. Se
		planea una misión de exploración, y para ello es necesario conocer
		ciertos datos. Entonces calcule:
		\begin{enumerate}
			\item Suponga que llevamos un péndulo simple, con una longitud
				de $l=3$\,m y con una masa de $5$\,kg. Calcule el peso y el
				período del péndulo sobre la superficie del planeta.
			\item la velocidad para que un objeto pueda escapar de dicho
				planeta.
			\item Un habitante de ese planeta que llegara a la Tierra, ¿podría
				ser considerado un superhéroe en función de su fuerza física,
				por ejemplo?
			\item Se está construyendo una nave en órbita terrestre, a una
				distancia de $r_i=1$\,UA del Sol. Despreciando el efecto de la
				gravedad terrestre, ¿cuál sería la velocidad que necesitamos
				darle a esta nave para que alcance dicho planeta con velocidad
				$v_f=0$?
		\end{enumerate}
		Datos útiles: $G = 6.67 \times 10^{-11}$\,N\,m$^2$\,kg$^{-2}$, $1$\,UA$
		= 1.5\times 10^{11}$\,m.  $M_\oplus=5.97\times 10^{24}$\,kg;
		$R_\oplus=6.37\times 10^{6}$\,m; $M_\odot=1.99\times 10^{30}$\,kg
	
	\item {\bf{Allá lejos, 3 puntos}}
		Una galaxia lejana se encuentra a $800$\,Mpc de distancia. Imagine que
		un observador situado en esa galaxia está observando a la nuestra, la
		Vía Láctea. Entonces.
		\begin{enumerate}
			\item ¿qué es lo que verá aquel exo-astrónomo? ¿qué nos alejamos de
				él?  ¿que nos acercamos a él? ¿estaría tentando a pensar que es
				el centro del Universo? ¿Por qué? Justifique claramente todos
				sus respuestas.
			\item A partir de la Ley de Hubble, calcule la velocidad entre
				aquella galaxia y la vía láctea.
			\item Calcule el valor de $z$ correspondiente, y estime cual será
				el valor de la longitud de onda observada $\lambda_o$ para la
				línea de absorción del hidrógeno $H_\alpha$, con
				$\lambda_e = 656.3$\,nm.
			\item En un futuro muy lejano, ¿usted cree que estaríamos más cerca
				o más lejos de él que ahora?
		\end{enumerate}
		Datos útiles: $H_0 = 67.3$\,km\,s$^{-1}$\,Mpc$^{-1}$;
\end{enumerate}
\end{document}

%%%%
